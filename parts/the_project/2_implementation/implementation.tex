\section{Regression}

\section{1D Convolutional Neural Network}

%%% The model %%%
\begin{code}
  \caption{1D Convolutional Neural Network Model}
  \label{code:1d_conv_net}
  
  \begin{minted}[linenos]{python}
    def create_model(segment_length, input_shape):
      model = Sequential()

      model.add(Reshape((segment_length, 1), input_shape=(input_shape,)))
      model.add(Conv1D(100, 10, activation='relu', input_shape=(segment_length, 1)))
      model.add(Conv1D(100, 10, activation='relu'))
      model.add(MaxPooling1D(2))
      model.add(Conv1D(160, 10, activation='relu'))
      model.add(Conv1D(160, 10, activation='relu'))
      model.add(GlobalAveragePooling1D())
      model.add(Dropout(0.5))
      model.add(Dense(2, activation='softmax'))

      return model
  \end{minted}
\end{code}

\begin{table}[]
  \begin{tabular}{|l|l|l|}
    \hline
    \multicolumn{1}{|l|}{\textbf{Layer (type)}}         & \multicolumn{1}{l|}{\textbf{Output Shape}} & \multicolumn{1}{l|}{\textbf{Param \#}} \\ \hline
    reshape (Reshape)                                   & (None, 480, 1)                             & 0                                      \\  
    conv1d (Conv1D)                                     & (None, 471, 100)                           & 1100                                   \\ 
    conv1d\_1 (Conv1D)                                  & (None, 462, 100)                           & 100100                                 \\ 
    max\_pooling1d (MaxPooling1D)                       & (None, 231, 100)                           & 0                                      \\ 
    conv1d\_2 (Conv1D)                                  & (None, 222, 160)                           & 160160                                 \\ 
    conv1d\_3 (Conv1D)                                  & (None, 213, 160)                           & 256160                                 \\ 
    global\_average\_pooling1d (GlobalAveragePooling1D) & (None, 160)                                & 0                                      \\ 
    dropout (Dropout)                                   & (None, 160)                                & 0                                      \\ 
    dense (Dense)                                       & (None, 2)                                  & 322                                    \\ 
    \hline
  \end{tabular}
  \caption{Model summary}
  \label{table:model_summary}
\end{table}

Following the tutorial on 1D CNNs \cite{1d_cnn}, we came up with this model \ref{code:1d_conv_net} after tweaking the parameters for our dataset.
 

\section{Optimizing the model}