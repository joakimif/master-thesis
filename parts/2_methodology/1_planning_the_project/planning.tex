\section{Goals}

Our goal in this thesis was to create machine learning models for three different tasks:

\begin{itemize}
  \item Classify whether a participant belongs to the \textbf{control} group or \textbf{condition} group.
  \item Classify a participant's depression class (by MADRS score).
  \item Predict a participant's MADRS score.
\end{itemize}

The models were going to be One-Dimensional Convolutional Neural Networks. We solved all these using each participant's activity data as input, and we created a model that, with few changes, could be used for all three goals. For the different goals, only the last few layers needed to change. 

For the first goal, classifying \textbf{control} or \textbf{condition} group, the output data was a matrix with two columns (\textbf{control}, \textbf{condition}), with a $1$ in one of the columns and a $0$ in the other for each row. 

However, first, we wanted to see if these problems were solvable with simple regression. The idea was to throw in the columns from the demographics dataset \ref{figure:demographics}. We did not expect much from this, as there are only 55 rows in the table. Anyone having a little bit of experience with machine learning will know that this is not nearly enough data. However, we wanted to do it regardless, and see how a simple and stupid model performed. Doing this, we established some benchmark for performance; the Convolutional Neural Network model had to \textit{at least} better than this one.

\subsection{Learning experiments}

We needed to learn more about convolutional neural networks, so we proceeded to implement an image recognition algorithm. We found a tutorial on how to make a 2D CNN for classifying cats and dogs from images \cite{2d_cnn} and thought it would be an excellent way to learn.

It was both a fun and informative experience in implementing this. Especially when we extended the script to allow an image URL to predict on, we could browse for images of cats and dogs on the Internet, and find out if the model could classify them (in most cases it did). We even tried inputting images of humans to the model for fun, to see if they looked more like a dog or a cat.

However as mentioned before, the activity measurements are of one dimension, so we could not use this network.

\begin{quote}
  \textit{A 1D CNN is very effective when you expect to derive interesting features from shorter (fixed-length) segments of the overall dataset and where the location of the feature within the segment is not of high relevance. This applies well to the analysis of time sequences of sensor data (such as data from gyroscopes or accelerometers).} \cite{1d_cnn}
\end{quote}

To learn more about one-dimensional convolutional neural networks, we followed a tutorial \cite{1d_cnn}, which used a dataset containing accelerometer data from a smartphone on the participant's waists. The goal here was to predict what a given person is doing at the time, given the accelerometer data for that time slice. What the given person is doing is one of the following:

\begin{itemize}
  \item Standing
  \item Walking
  \item Jogging
  \item Sitting
  \item Upstairs
  \item Downstairs
\end{itemize}

As we followed the tutorial and implemented the model, we learned a lot about how one-dimensional convolutional neural networks work and how we should structure our data to make it work for our dataset.

We also got some ideas about where our dataset could provide more data. What if the dataset containing the current mental state of the bipolar patient? Then someone could make some automated system that always can tell a patient whether they are not depressed, manic or depressive. However data collection for this kind of task would be difficult because we cannot always know what the patient thinks, nor does the patient. The tutorial dataset is different because it is easy to differentiate physical states of the body like standing or walking.

\section{The dataset}

\begin{figure}
    \begin{center}
        \includegraphics[height=3.5cm]{img/demographics.png}
        \caption{5 first rows in demographic dataset (scores.csv).}
        \label{figure:demographics}
    \end{center}
\end{figure}

\begin{figure}
    \begin{center}
        \includegraphics[height=3.5cm]{img/participant.png}
        \caption{5 first rows in activity measurements dataset for one participant (condition\_1.csv).}
        \label{figure:participant_activity}
    \end{center}
\end{figure}

The dataset we will use in this project was collected for another study for motor activity in schizophrenia and major depression. With the data about schizophrenia stripped out, the dataset is sufficient for this thesis. It contains activity level data for 23 bipolar/unipolar patient, and 32 non-depressed contributors. From now on, we will refer to the bipolar/unipolar group as the \textit{condition group}, and the non-depressed group as the \textit{control group}. The dataset details \cite{dataset_details} also follows this convention.

The dataset is in two parts. One part includes the demographics of each participant \ref{figure:demographics}, where the fields are \cite{dataset_details}:

\begin{itemize}
    \item \textbf{number}: a unique id for each participant
    \item \textbf{days}: number of days of activity data collection 
    \item \textbf{gender}: 1 for female and 2 for male
    \item \textbf{age}: participant's age (grouped by four years)
    \item \textbf{afftype}: affliction type, where 1 is for bipolar type II, 2 equals unipolar depressive, and 3 for participants with bipolar type I
    \item \textbf{melanch}: 1 means a participant has melancholia, 2 means no melancholia
    \item \textbf{inpatient}: whether the patient is inpatient (1) or outpatient (2)
    \item \textbf{edu}: participant's education in years (grouped by four years)
    \item \textbf{marriage}: married or cohabiting (1) or single (2)
    \item \textbf{work}: whether the participant is working or studying (1) or not (2)
    \item \textbf{madrs1}: MADRS score before activity measurement started
    \item \textbf{madrs2}: MADRS score after activity measurements ended
\end{itemize}

The second part includes sensor data about the condition group and control group, as one file for each participant \ref{figure:participant_activity}. These files are in two folders for the two groups (control/condition) respectively, and one file for each person inside the folders (filename is "GROUP\_X.csv" where X is their id and GROUP is either condition or control. Inside the files, there is a list of activity measurements for every minute of the data collection period.

\section{Data Preprocessing}

\subsection{Input data}

As we said before, we wanted the input data to be the same for each goal, since we wanted to use a similar model on all three.
The tutorial \cite{1d_cnn} sliced up the measurements with overlap, and labeled the slices. We also did this.

We created a list where for each participant in the demographics table \ref{figure:demographics}, measurements for $N$ hours were grouped. 
Another choice we learned from the tutorial was to overlap the sequences, so we made the next group of $N$ hours start \textit{$M$} hours after, 
and not \textit{$N$} hours after the group before, as one might think. When this list was complete with sequences from all participants, we 
had to \textbf{reshape} it so that it could fit into a neural network. We ended up with a feature list (which we called \textbf{segments}), 
where each element was a list of activity measurements for 4 hours: 

\textbf{segments[i] = [[0], [143], [0], [20], [166], [160], [306], [277], [439], ...]}

\subsection{Output data}

A second list was created simultaneously, where the value here was different for each goal. 

\subsubsection{Classifying participant group}

For classifying control/condition group, this list was built to contain the values \textbf{0} or \textbf{1} for the labels \textbf{[CONTROL]} and \textbf{[CONDITION]}, which was chosen according to the group the participants were in. Using a helper function from Keras, \textbf{to\_categorical}, we transformed this list of labels into the matrix we described. We had to transform the values to a categorical matrix in order to be able to select a \textit{category} for classification. This list, which we called \textbf{labels}, looked like this: 

\textbf{labels[i] = [0, 1]}

\noindent Meaning that segment $i$ is labeled as \textbf{[CONDITION]}.

\subsubsection{Depression Classes}
Here we wanted to classify which depression class a participant belonged to, and as described in the earlier, we divide MADRS scores into some cutoff-points, which we used as \textit{classes} in our classification:

\begin{itemize}
  \item 0-6: normal
  \item 7-19: mild depression
  \item 20-34: moderate depression
  \item 34-60: severe depression
\end{itemize}

So instead of labelling the segments as \textbf{[CONTROL]} or \textbf{[CONDITION]}, we labeled them as 0, 1 or 2 (we ignored 3 as there are no participants with MADRS scores this high). One element in \textbf{labels} after making it a categorical matrix looked like this:

\textbf{labels[i] = [0, 1, 0]}

\noindent Meaning that segment $i$ is labelled as \textbf{[MILD DEPRESSION]}.

\subsubsection{Predicting MADRS Score}

Instead of classifying one of three or one of two classes which we have done earlier, this time we wanted to predict the actual MADRS score value. Creating the output data for this goal is easier; we appended the MADRS score for the corresponding participant to the list. For example:

\textbf{scores[i] = [18]}

\subsection{Creating Segments and Labels}
\textit{create\_segments\_and\_labels()} \ref{code:reading_dataset} was responsible of creating the data that is sent into the neural network. We started by defining a \textit{segment length} $L$, which is how much data (minutes) we want inside each segment. We will experiment with the value of $L$, but let's say we use segments of 4 hours at a time ($L=4*60=240$). Next, we need a value for how many values to step after each iteration, $S$. We will keep this value at one hour, meaning $S=60$. Between the different goals, this function will only be different in how it determines the output values. The code in the Source Code section \ref{code:reading_dataset} is simplified to only generate input and output for classifying control/condition groups (goal one).

\begin{table}
  \begin{center}
    \begin{tabular}{| l | l |}
      \hline
      \textbf{Control group} & \textbf{Condition group}  \\ \hline
      0                    &  1                \\ \hline
      1                    &  0                \\ \hline
      0                    &  1                \\ \hline
      1                    &  0                \\ \hline
      1                    &  0                \\ \hline
      0                    &  1                \\ \hline
      0                    &  1                \\ \hline
      1                    &  0                \\ \hline
    \end{tabular}
    \caption{Categorical Labels. A 0 and a 1 (first row) means that the participant is in the condition group.}
    \label{table:categorical_labels}
  \end{center}
\end{table}

\begin{itemize}
      \item First we read the \textit{global} dataset, where we find each participant and whether they are bipolar or not. As there is no \textit{afftype} value
            for non-bipolar participants, we simply set this to 0. This is fine because the other possible values are 1, 2 and 3.
      \item Then we iterate over all participant activity data files:

      \begin{itemize}
        \item Append a \textbf{segment} that is of length $L$ to the list of segments (using default parameters in the
              \textit{create\_segments\_and\_labels} function \ref{code:reading_dataset}).
        \item Append the target value for the current goal, so:
          \begin{itemize}
                \item Append a $1$ or $0$ for classifying control/condition group.
                \item Append a $0$, $1$ or $2$ for classifying depression class \\(normal/mild/moderate).
                \item Append the MADRS score (after measurement period) when the goal is to predict MADRS score.
          \end{itemize}  
        \item Skip $S$ indexes, and repeat until we have added all segments.
      \end{itemize}

      \item Make the list of labels into a \textit{categorical} 2D matrix \ref{table:categorical_labels} with a \textbf{1} in only one of the columns,
            instead of a single-dimensional list.
            This is only needed in the first two goals, for the \textbf{softmax} activation function.
      \item Also we need the list of segments to be restructured. We do this with the \textbf{reshape} function, 
            and after this, the data is ready to be passed into the neural network.
\end{itemize}

\section{Performance}

\input{parts/2_methodology/1_planning_the_project/performance.tex}