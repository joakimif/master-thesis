Teaching machines to learn patterns in data is very common these days, and it has a broad spectrum of applications everywhere. Sensors like smart-watches are getting more functionality each year, and more and more people buy them. Passing data from the watches, for example, activity or heart rate to machine learning algorithms, can generate significant results within many fields. Mental health is an example of a field where computer-generated predictions can be helpful to gain knowledge about patients. For example, machine learning can help predict that someone has a specific type of mental disorder.

In this thesis, we present applied machine learning to detect depression. The dataset (collected for another study about behavioral patterns in schizophrenia vs. major depression) contains motor activity measurements for each minute in the measured period for each participant. Three machine learning models are trained to fit time-sliced segments of these measurements. The first model classifies participants into  condition group (depressed) and control group (non-depressed). We trained another model to classify the depression level of participants (\emph{normal}, \emph{mild} or \emph{moderate}). Finally, we trained a model that predicts MADRS scores. We evaluate the performance of classification models using leave one participant out validation as a technique, in which we achieved an average F1-score of 0.70 for detecting control/condition group and 0.30 for detecting the depression levels. The MADRS score prediction resulted in a mean squared error of approximately 4.0. 

These performance scores are promising, but not good enough to be used in the real world. However, not much more work is needed for the first model if we apply it to a dataset with more participants.