\section{Real world applications}

The field of mental health is still at an early stage when it comes to machine learning. Several researchers have done research comparing different approaches and algorithms to see which of them work better than others for detecting mental diagnoses. The results of these research papers are promising, but further work is necessary as the goal is to trust the decisions of machine learning someday. 

The performance was better when training with data from all patients (accuracy scores above 0.99), which tells us that the models were able to pick up features successfully. However, we learned from the \textit{leave one participant out} experiment that the difference between people's activity behavior is too significant for those in the condition group (F1-score was only 0.64). 

Because of this, personal activity datasets is possibly a better way of using convolutional neural networks. If we continuously save motor activity for each participant and train a model specific to each participant, then the network can learn all there is to know about one person and for example, be used to detect bipolar state changes. Combined with the research of Grünerbl, A. et al., where they managed to identify state changes with an accuracy of 0.76 using phone call logs and microphone data \cite{grunerbl_smartphone_bipolar}, development of more accurate detection systems could be possible.

\subsection{Privacy}
When research in the field of MHMS is where we want it, and starts to get used in hospitals and institutions, we need to take the storing procedures of the data into account. Inputs and outputs of these systems are sensitive data, and we should treat them in the same way as any other health record. Unauthorized access to this kind of data can have severe consequences for patients, and also for whoever is responsible for storing it. 

Machine learning technology is getting better and better with improved hardware and continuous research. However, the use cases of it are not only those with a legitimate purpose. We believe sophisticated cyber attackers are going to start using Artificial Intelligence for their cause, which makes the task of keeping data safe sound more complicated than it is today.

\subsection{Ethical concerns}
A question is whether we should trust a machine when it predicts that a person has a mental illness. For as long as we have human doctors, we think that machine learning based decisions about mental health should not be the only factor of a diagnosis. Until the day machines take over this kind of work, we think doctors should use such predictions as a tool to decrease their workload, so that they can help more people. Machine learning based medical assistant tools also make the difference between each doctor/institution less significant, as these tools will most probably contain more data than each doctor's individual experience.  