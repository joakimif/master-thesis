Teaching machines to learn patterns in data is very common these days and has a broad spectrum of applications everywhere. Sensors like smart-watches are getting more functionality each year, and more and more people buy them. Passing this kind of data to machine learning algorithms can generate significant results within many fields. Mental health is an example of a field where computer-generated predictions can be helpful.

In this thesis, we present applied machine learning to detect depression. The dataset contains motor activity measurements for each minute in the measured period for each participant, and the machine learning models are trained to fit time-sliced segments of these measurements. The first model detects whether a participant most likely belongs to the condition group (depressed) or the control group (non-depressed). We trained another model to classify the depression level of participants (divisions of MADRS scores), and finally, a model that predicts MADRS scores. We evaluate the performance using leave one participant out validation as a technique, in which we achieved an average F1-score of 0.70 for detecting control/condition group and 0.30 for detecting the depression levels (\emph{normal}, \emph{mild} and \emph{moderate}). MADRS score prediction resulted in a mean squared error of approximately 4.0.