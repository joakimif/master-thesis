\section{Convolutional neural networks for mental health detection}
We have evaluated how convolutional neural networks performed on motor activity measurements from bipolar and unipolar depressed patients (condition group) together with non-depressed control participants. We achieved our goals of creating classification models for detecting whether a participant belongs to the control group or the condition group, the depression class of a participant (not depressed, mild depression or moderately depressed), and finally a prediction model for estimating the MADRS score of participants. 

We did not leave one participant out on the last model, because it would require too much time. The model was built for predicting the MADRS score of participants and had to be trained for at least 1500 epochs before the mean squared error was acceptable (see figure \ref{figure:madrs_prediction_history}). Several hours of training were needed, and multiplying that with 55 (one training session for each participant), the experiment would take days. We consider this to be a flaw in our work and it makes the model less trustworthy. 

\subsection{Limited number participants}
The dataset consists of only 55 participants, which is very limited. It was a good set of data for us in our thesis, but in order to use it in the real world, more participants is a requirement. Even when splitting the data into multiple parts to train and test on different data and avoid overfitting, it is difficult to know how the model would perform on an arbitrary person. Measurements from more people and different ethnicities and age groups around the world would significantly increase the viability of the dataset. We suggest collecting more data before applying this research anywhere in the real world.
 
We think that the number of participants is the main issue that made the classifier only decent when leaving one participant out. The difference between people is far too big for our convolutional network to handle with a dataset containing such few participants. With activity measurements from more people, maybe the network would pick up and learn more similarities between people within the same \textit{group}, and be able to classify with higher performance. 

\subsection{Input data and hyper-parameters}
Detecting control/condition group was our first experiment, and we found that for our dataset the optimal amount of data inside each segment was 2880 (48 hours in minutes). Following our calculations, the optimal segment length was 5760 (96 hours in minutes) for the next experiment where the goal was to detect depression classes. When we predicted MADRS scores, the optimal segment length was 2880 once again. 

We are aware of the fact that our calculations only apply to the specific segment lengths tested, and testing the performance using longer segments may have resulted with something else. However, we did not prioritize to do more of such experiments, as the training time would increase too much if we were to continue using the hardware that we used in our other experiments. 

It is also possible to tweak hyper-parameters more than we ended up doing. In the first two experiments, we did not touch anything else than the length of the input segments and the number of epochs. It was first in the prediction model that we started experimenting with different optimizers and the learning rate parameter. 

We suggest that future research experiment more with hyper-parameters. However, it is difficult to know whether tweaking hyper-parameters is a potential fix to the poor performance in the leave one participant out experiment. We assume it would not help that much because the training accuracy always ended up above 0.99. 

\subsection{Compared to earlier research}
Earlier research on the topic of classifying depression is different from this thesis, as most of them compare how different types of machine learning performs classification. In contrast, we focused on one specific type of supervised machine learning: convolutional neural networks, which we used to achieve different goals. Instead of creating baseline models and comparing our results to them, we compare our work to what Garcia-Ceja, E. et al. achieved on the same dataset (we will only focus on our first goal in this comparison, as it is the only experiment they also performed). 

Overall, we achieved an F1-score of 0.70 for classifying control/condition group, which is slightly better than the F1-scores from the research of Garcia-Ceja, E. et al. without oversampling. They achieved 0.66 for the deep neural network and 0.67 for the random forest \cite{GarciaCeja2018_classification_bipolar}. When using SMOTE as a technique for generating more data, they increased their random forest F1-score to 0.73. A suggestion for future experiments is to attempt using the same sampling strategies as Garcia-Ceja, E. et al. on the data passed into our convolutional network, and check if the performance gets any better. 

Our convolutional network performed a little bit better than the random forest and the deep neural network of Garcia-Ceja, E. et al., but it was not as good as we hoped considering the results of other experiments. The question of whether convolutional networks as a machine learning approach is a reliable option to use in mental health remains unanswered, as our results were only promising and not anywhere close to perfect. Nonetheless, our opinion is that it was a step in the right direction, and we hope researchers continue to explore this type of machine learning.

Garcia-Ceja, E. et al. also suggested future research to explore classification based on the MADRS scale, which we implemented as our second goal. The results were similar to our first experiment; not so good performance overall but able to classify most non-depressed participants correctly. 

\section{Real world applications}

The field of mental health is still at an early stage when it comes to machine learning. Several researchers have done research comparing different approaches and algorithms to see which of them work better than others for detecting mental diagnoses. The results of these research papers are promising, but further work is necessary as the goal is to trust the decisions of machine learning someday. 

The performance was better when training with data from all patients (accuracy scores above 0.99), which tells us that the models were able to pick up features successfully. However, we learned from the \textit{leave one participant out} experiment that the difference between people's activity behavior is too significant for those in the condition group (F1-score was only 0.64). 

Because of this, personal activity datasets is possibly a better way of using convolutional neural networks. If we continuously save motor activity for each participant and train a model specific to each participant, then the network can learn all there is to know about one person and for example, be used to detect bipolar state changes. Combined with the research of Grünerbl, A. et al., where they managed to identify state changes with an accuracy of 0.76 using phone call logs and microphone data \cite{grunerbl_smartphone_bipolar}, development of more accurate detection systems could be possible.

\subsection{Privacy}
When research in the field of MHMS is where we want it, and starts to get used in hospitals and institutions, we need to take the storing procedures of the data into account. Inputs and outputs of these systems are sensitive data, and we should treat them in the same way as any other health record. Unauthorized access to this kind of data can have severe consequences for patients, and also for whoever is responsible for storing it. 

Machine learning technology is getting better and better with improved hardware and continuous research. However, the use cases of it are not only those with a legitimate purpose. We believe sophisticated cyber attackers are going to start using Artificial Intelligence for their cause, which makes the task of keeping data safe sound more complicated than it is today.

\subsection{Ethical concerns}
A question is whether we should trust a machine when it predicts that a person has a mental illness. For as long as we have human doctors, we think that machine learning based decisions about mental health should not be the only factor of a diagnosis. Until the day machines take over this kind of work, we think doctors should use such predictions as a tool to decrease their workload, so that they can help more people. Machine learning based medical assistant tools also make the difference between each doctor/institution less significant, as these tools will most probably contain more data than each doctor's individual experience.  