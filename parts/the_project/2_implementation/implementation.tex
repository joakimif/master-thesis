\section{Regression}

\section{1D Convolutional Neural Network}

%%% The model %%%
\begin{code}
  \caption{1D Convolutional Neural Network Model}
  \label{code:1d_conv_net}
  
  \begin{minted}[linenos]{python}
    def create_model(segment_length, input_shape):
      model = Sequential()

      model.add(Reshape((segment_length, 1), input_shape=(input_shape,)))
      model.add(Conv1D(100, 10, activation='relu', input_shape=(segment_length, 1)))
      model.add(Conv1D(100, 10, activation='relu'))
      model.add(MaxPooling1D(2))
      model.add(Conv1D(160, 10, activation='relu'))
      model.add(Conv1D(160, 10, activation='relu'))
      model.add(GlobalAveragePooling1D())
      model.add(Dropout(0.5))
      model.add(Dense(2, activation='softmax'))

      return model
  \end{minted}
\end{code}

Following the tutorial on 1D CNNs \cite{1d_cnn}, we came up with this model \ref{code:1d_conv_net} after tweaking the parameters for our dataset.
 
\begin{itemize}
  \item First, the input data needs to be reshaped
\end{itemize}
\section{Optimizing the model}