In this thesis, we have presented our implementation of convolutional neural networks that use the \textit{Depresjon} dataset, which is a minute by minute log of motor activity for each participant. We created three models, and the first can classify with promising performance (F1-score of 0.70) whether a participant belongs to the condition group (bipolar and unipolar depressed patients) or the control group (healthy participants). 

Another convolutional model, with the same data as input, detects one of three different levels of depression based on MADRS scores. We labeled participants in the control group as non-depressed and divided participants in the condition group into mildly depressed (MADRS between 7 and 19) and moderately depressed (MADRS between 20 and 34). Then we trained the model to detect this label for participants. We achieved an overall F1-score of 0.3 for this goal, which has a significant room for improvement.  

Our third and last model is a prediction model that aims to predict the MADRS score of participants, again using the same motor activity dataset as input. We did not leave one participant out to test on as we did in the other goals, as we did not have the computing hardware to train the model 55 times. Instead, we trained one model for 2700 epochs and achieved a mean squared error of approximately 4.0. 

We found that our models performed almost flawlessly (F1-scores above 0.99 for classification and mean squared error of approximately 4.0) when training on data that includes the tested participants. There is a significant difference between these results and the experiments where we left one participant out of training. It may indicate that convolutional networks are more useful for personal datasets. 

We presented both a medical and technical background in chapter 2, in addition to mentioning related work to our thesis. We described both bipolar disorder and Montgomery-Åsberg Depression Rating Scale as part of the medical background. It was essential to gain knowledge about these topics because as students in Computer Science, our knowledge was limited. The technical background contains an introduction to machine learning, which is both theoretical and practical. In both the examples in this part and our model implementations, we used Keras, a machine learning framework in Python, because of its simplicity. 

In the next chapter, we described the dataset, our primary goals and how we structured our data so that a convolutional neural network would be able to learn from it. The input data consisted of time-sliced segments mapped to the corresponding element in the output data. The output data contained one of three target values (from the demographics part of the dataset), which depended on the goal. We also touched upon different performance metrics that we used later in classification experiments.

In chapter 4, we introduced a small regression test to see if we could learn something from any column in the demographics dataset. As we expected, only the MADRS score had a relation to whether a participant was in the control or condition group. Then we proceeded to implement our convolutional neural networks. We built three different models for our three primary goals (the first two models were very similar, as only the number of output possibilities changed - depression level instead of control/condition). The last model was different in several layers, as it was built to predict MADRS values. 

We presented the training results in chapter 5. For each goal (including linear regression), we described the hyper-parameters we used when training the models. For the convolutional neural networks, we started out finding the optimal segment length for the input data before we trained the models. Then we calculated performance scores for the trained models and did cross-validation to ensure the consistency of the models. For the classifiers, we also did a final experiment where we left participants out one by one before training the models, then tested the models' capabilities to detect that particular participant.

We discussed our work in chapter 6 and came up with several issues that made our models perform the way they did. The amount of participants is limited, which is the main issue that we think reduces the detection capabilities of the models when it comes to testing on completely untouched participants. Further improvements can be experimenting more with the segment lengths (input data) and other hyper-parameters. 

We compared our work to the research Garcia-Ceja, E. et al. performed on the same dataset and found that the difference between their decision tree and neural network and our convolutional network was not that significant as we thought it would be. We discussed real-world use cases for convolutional neural networks in mental health. The excellent performance from when we included data from all participants in training gave us the idea that this kind of machine learning would perform better in detection within personal datasets. We also discussed a few ethical concerns that arise in the field of Mental Health Monitoring Systems. <More>

\section{Future work}
As stated earlier, there are several weaknesses in both the dataset and our convolutional neural network implementation. We split suggested future work in three parts; continued exploration of convolutional neural networks, optimizing datasets for mental health purposes, and other machine learning approaches. 

First and foremost, leaving one participant out of training, as we did in for the first two goals, is something we want future research to also use as a performance evaluation. It is arguably the most accurate way of checking \textit{real world} consistency of the trained model. However, it may also be time-consuming depending on the complexity of the model and input data (as previously stated about our MADRS prediction model). For those models, we suggest leaving multiple participants out instead of one (K-fold cross-validation on participants).

We did not find that convolutional networks were any better than decision trees on the kind of data that we provided to the models. Because the complexity of a CNN is higher, we want future researchers to make deeper CNN models and experiment with hyper-parameters. 

Researchers have experimented with different kinds of data in the field of mental health. Motor activity data \cite{obrien_depression, GarciaCeja2018_classification_bipolar}, Instagram images \cite{instagram_depression}, Twitter posts \cite{twitter_depression}, phone call logs \cite{faurholt_smartphone_bipolar, grunerbl_smartphone_bipolar}, text messages and voice data from microphones \cite{grunerbl_smartphone_bipolar} are examples of data used in earlier research. Diversity in the type of data is something we want to see in future research as well. Specifically for motor activity data, we suggest including more participants, which we think will improve the performance significantly.

It can be useful to explore and compare different machine learning approaches with convolutional networks. Garcia-Ceja, E. et al. mentioned classification algorithms such as recurrent neural networks and hidden Markov models to be investigated in future research \cite{GarciaCeja2018_classification_bipolar}. We did not use these algorithms in this thesis, and therefore leave them as a suggestion to future researchers. 