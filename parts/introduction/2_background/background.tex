%%%%%% BACKGROUND %%%%%%
\newpage
\section{Bipolar disorder}

Bipolar disorder is the disorder where you experience extreme mood swings. One day you can feel amazing and 
everything is fine, but the next day you feel like you don't belong anywhere in this universe. 
Mood swings in general is not something that you should be concerned about. It is however the extreme cases where 
your mind turns 180 degrees from day to day that is the main symptom of bipolar disorder. 
There is not really a specific type of people that get this; they can be of any age and any gender, 
but most people that suffer from it find out (by having an experience or episode) around age 25 \cite{bipolar_statistics}. 

When talking about bipolar disorder, we often separate between the states \emph{normal}, \emph{mania} 
and \emph{depression}. The last two are the states we usually talk about, since a normal state isn't that interesting. 
These two states are very different, but they have some similarities, for example sleeping problems. 

When a bipolar person is in a manic state, he/she may do things that they never would have intended doing, 
like spending a lot of money on items they really don't need, or abusing drugs/alcohol. 
They may also feel really excited or powerful \cite{bipolar_overview}. 

A bipolar patient is in a depressive state when he or she is in a bad mood swing. They can stop doing everything 
they usually like to do, and lie down in bed all day with no motivation to do anything useful. 
They may feel useless and that they don't belong here, or being guilty of something they may or may not have done. 
In some cases, a depression may even end up with suicidality, where the person either just thinks of death, 
or actually attempt suicide (actually 20\% of people diagnosed with bipolarity commit suicide \cite{bipolar_statistics}). 

The frequency of these symptoms can vary. One year they can have these mood swings every day for several weeks at the time, 
and the next they get them less frequent, like once every month. 

We also separate between bipolar disorder type I and II, with the main difference being that the manic episodes 
are way more aggressive in type I \cite{bipolar_types}.

Statistics say that bipolarity is genetically inheritable, with 23\% chance of getting a child with bipolar disorder 
if one parent is bipolar, and 66\% if both parents are \cite{bipolar_statistics}. 

%% Use stories from book?

\newpage
\section{Machine learning}
Machine learning is the field of computer science where you basically throw a lot of data into an algorithm
and expect it to give you answers to whatever you prefer, with as little work as possible. 
This was not the case in the early days of the technology, but nowadays it is a lot easier with all the diffent 
frameworks and tools available.

Machine learning is a great and almost "magical" technology, but only if you do it right.
First you need to have enough data to feed into the algorithm, and to be efficient when training the model on a 
large dataset, which you need to be if you want your result quickly, you need good hardware. You can get away with a 
decent CPU if you just want to test it out on a small dataset, but if you really want to do machine learning,
then you need a good GPU. The reason why GPUs are so much better than CPUs on this specific task, 
is because the CPUs are designed for flexibility and general computing workloads. The GPUs on the other hand, 
are designed to do the same instructions over and over again in parallel. This makes GPUs a lot more efficient for 
machine learning, and especially for deep neural networks \cite{cpu_vs_gpu_ml}. 

Now how do you do the actual machine learning? Well there are many diffent approaches to this, which I will come back to 
in the next sections. 
For example in the programming language \textbf{Python} you have the 
framework \textbf{TensorFlow}, which allows you to build models in a simple way, and also execute the training. 
You actually might not even have to implement the model yourself, because so many models have been implemented already. 

Another framework for Python is \textbf{Keras}. On their documentation website \cite{keras_docs}, they see their 
framework as \blockquote{A high-level neural networks API, written in Python and capable of running on top of TensorFlow, 
CNTK or Teano}. This makes the programming of the model even easier, and this is the framework I will use 
on top of TensorFlow for the deep learning programming parts of my thesis. 

So, as long as you know your theory, and can decide which machine learning model to use (and you find a good 
implementation of that model), you really just have to make the dataset ready. This is the boring and tedious part 
of machine learning, but it has to be done in order for making it possible to train the model on it. 



\newpage
\section{Machine learning strategies}
\subsection{Supervised learning}
\subsection{Unsupervised learning}
\subsection{Semi-supervised learning}

\newpage
\section{Machine learning approaches}
\subsection{Decision tree learning}
\subsection{Reinforcement learning}
\subsection{Neural networks}
\begin{itemize}
    \item General idea
    \item Deep learning
\end{itemize}

%% More approaches?
\newpage
\section{How can machine learning help}
How I can solve the problem described in the beginning of this chapter...

\newpage
\section{Related work}
List related work and discuss...

\newpage
\section{Challenges}
Describe some challenges...

\newpage
\section{Ethical concerns}
Describe some ethical concerns...
