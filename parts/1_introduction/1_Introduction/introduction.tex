\section{Motivation}

We are using Artificial Intelligence (AI) everywhere these days. Everyone is talking about how self-driving cars will change the world. Every smartphone company includes AI in their phones, striving to make the day of the user effortless. Thanks to Machine Learning or more specifically Deep Learning, AI can already beat humans in most games and do repetitive tasks for us. The concept of AI has been around for a long time (more than 80 years by among others Alan Turing \cite{turing1938}), but until 2012 Deep Learning was not a valid form of AI \cite{topol2019}. Today Deep Learning is widely used by the largest companies to learn all patterns possible about their customers. 

Research on the usage of AI in the medical field is particularly interesting, as the results can be life-changing. Instead of getting diagnosed only by a single doctor, a global AI can influence his or her decisions. The difference between independent doctors worldwide would not matter as much, and with high accuracy, doctors instantly would know a lot about the patient. 

Three hundred and fifty million people globally are fighting the burden of depression \cite{burden_of_depression}, which in a lot of the cases does not end well for them. AI has the potential here to help with diagnoses, help in the process of prescribing the right medicine to patients and predict that a person is likely to attempt suicide \cite{topol2019}. 

There is no limit to the usefulness of machine learning in the medical field, and it can undoubtedly help in the mental health field too. For example, say bipolar patients had a device that measured their heart rate among other things 24 hours a day could feed the data into a neural network that gave the user's bipolar state as output. That would be useful for both the patients and doctors/nurses. Another use case could be if medical institutions knew in advance how many new bipolar patients to expect the next day. Using machine learning in this field of study could help many people get through their depression or mania, and potentially get rid of the condition altogether.

\section{Thesis Overview}
The main goal of the thesis is to apply convolutional neural networks in the field of mental health and find whether motor activity measurements are valid as input data to detect depression. We divide our goal into three objectives which we implement convolutional neural network models in order to satisfy. We provide a walkthrough of the whole process, from data preprocessing to predictions and classifications from the trained neural networks which we use to determine, for each objective, the final detected outcome of tested participants.

\textbf{The first objective} is to build a convolutional neural network to detect depression. We do this by dividing participants into a control group (non-depressed) and a condition group (unipolar depression or bipolar disorder type 1 or 2). The neural network should be able to detect (classify) whether a participant (where the motor activity measurements are unseen by the model while training) belongs to the control group or the condition group. We view this objective as the most relevant because it is the only objective that is comparable to earlier research.

\textbf{The second objective} is to build a convolutional neural network model to detect \textit{levels} of depression. We divide the levels of depression into the following groups based on the Montgomery-Åsberg Depression Rating Scale (MADRS): no depression, mild, moderate and severe depression. The neural network should be able to classify the depression level of an arbitrary participant.

\textbf{Our third and last objective} is building a convolutional neural network that predicts the MADRS score of participants. We evaluate this neural network by calculating the \textit{mean squared error} of the predictions.

We evaluate the performance of classification models using a \textit{leave one participant out} technique combined with majority voting. The reported performance with an average F1-score of 0.70 (first objective), 0.30 (second objective) and a mean squared error of approximately 4.0 (third objective) indicates that motor activity measurements can give useful information about mental health issues, and is a relevant topic for further exploration.

\subsection{Thesis outline}
In \textbf{chapter \ref{chapter:background}} we provide background information about mental health and machine learning. In mental health, we describe the topics of bipolar disorder and how the Montgomery-Åsberg Depression Rating Scale can tell patients how depressed they are. Then we introduce important topics within machine learning, and how easy it is to get started writing models in Keras, a machine learning framework for Python. Related work is the last section in chapter 2, where we mention several papers related to the topics of mental health and machine learning. \\

\noindent \textbf{Chapter \ref{chapter:planning}} is where we describe the objectives that we want to achieve in the thesis. We present the dataset we use to achieve those objectives, and also how we structure and preprocess the data so that it can be used to train models. Performance metrics that we use later in order to evaluate a trained model is another topic we define in the chapter.\\

\noindent \textbf{Chapter \ref{chapter:models}} contains a description of the neural network model for each objective in addition to a simple linear regression model. We describe all layers in the models and include the source code used to create them in the machine learning framework Keras. \\

\noindent In \textbf{chapter \ref{chapter:training}} we walk through how we trained the models in order to reach our objectives, then evaluated them with metrics described in chapter 3. \\

\noindent \textbf{Chapter \ref{chapter:discussion}} is where we discuss whether convolutional neural networks can be used to classify and predict mental health issues. We also suggest real-world applications of our work, focusing on the pros and cons of creating and trusting an AI for mental health diagnoses. \\

\noindent \textbf{Chapter \ref{chapter:conclusion}} contains our conclusion and a summary of the chapters, and we give directions for future research.

