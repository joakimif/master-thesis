In this thesis, we have presented our implementation of convolutional neural networks that use the \textit{Depresjon} dataset, which is a minute by minute log of motor activity for each participant. We created three models, and the first can classify with promising performance (F1-score of 0.64) whether a participant belongs to the condition group (bipolar and unipolar depressed patients) or the control group (healthy participants). 

Another convolutional model, with the same data as input, detects one of three different levels of depression based on MADRS scores. <F1?>

Our third and last model is a prediction model that aims to predict the MADRS score of participants, again using the same motor activity dataset as input. <F1?>

We found that our models perform almost flawlessly (F1-scores above 99\% for classification and mean squared error of approximately 4) when training on data that includes the tested participants. The difference between these results and the experiments where we left one participant out of training is interesting, and it may indicate that convolutional networks are more useful for personal datasets. 

We presented both a medical and technical background in chapter 2, in addition to mentioning related work to our thesis. We described both bipolar disorder and Montgomery-Åsberg Depression Rating Scale as part of the medical background. It was essential to gain knowledge about these topics because as students in Computer Science, our knowledge was limited. The technical background contains an introduction to machine learning, which is both theoretical and practical. In both the examples in this part and our model implementations, we used Keras, a machine learning framework in Python, because of its simplicity. 

In the next chapter, we described the dataset, our primary goals and how we structured our data so that a convolutional neural network would be able to learn from it. The input data consisted of time-sliced segments mapped to the corresponding element in the output data. The output data contained one of three target values (from the demographics part of the dataset), which depended on the goal. We also touched upon different performance metrics that we used later in classification experiments.

In chapter 4, we introduced a small regression test to see if we could learn something from any column in the demographics dataset. As we expected, only the MADRS score had a relation to whether a participant was in the control or condition group. Then we proceeded to implement our convolutional neural networks. We built three different models for our three primary goals (the first two models were very similar, as only the number of output possibilities changed - depression level instead of control/condition). The last model had to be different in several layers, as it was built to predict MADRS values. 