\section{Convolutional Neural Networks for Mental Health Classification/Prediction}

In this thesis, we evaluated how convolutional neural networks performed on activity measurements from bipolar and unipolar depressed patients (condition group) together with non-depressed control participants. We achieved our goals of creating classification models for detecting whether a participant belongs to the control group or the condition group, the depression class of a participant (not depressed, mild depression or moderately depressed), and finally a prediction model for estimating the MADRS score of participants. 

First and foremost, the dataset consisted of only 55 participants, which is very limited. It was a good set of data for us in our thesis, but in order to use it in the real world, more participants is a requirement. Even when splitting the data into multiple parts to train and test on different data and avoid overfitting, it is difficult to know how the model would perform on an arbitrary person. Measurements from more people and different ethnicities and age groups around the world would significantly increase the viability of the dataset. We suggest collecting more data before applying this research anywhere in the real world.

Detecting control/condition group was our first experiment, and we found that for our dataset the optimal amount of data inside each segment was 2880 (48 hours in minutes). Following our calculations, the optimal segment length was 5760 (96 hours in minutes) for the next experiment where the goal was to detect depression classes. When we predicted MADRS scores, the optimal segment length was 2880 once again. 

We are aware of the fact that our calculations only apply to the specific segment lengths tested, and testing the performance using longer segments may have resulted with something else. However, we did not prioritize to do more of such experiments, as the training time would increase too much if we were to continue using the hardware that we used in our other experiments. 

It is also possible to tweak hyper-parameters more than we ended up doing. In the first two experiments, we did not touch anything else than the length of the input segments and the number of epochs. It was first in the prediction model that we started experimenting with different optimizers and the learning rate parameter. We suggest that future research (preferably with more participants) experiment more with hyper-parameters.  

Earlier research on the topic of classifying depression is different from this thesis, as most of them compare how different types of machine learning performs classification. In contrast, we focused on convolutional neural networks, which is one specific type of supervised machine learning. Training a network of this type on such small dataset yielded promising results. We achieved an F1-score of 0.997 for classifying control/condition group, and the research performed by Garcia-Ceja, E. et al. only resulted in F1-scores around 0.7 on the same dataset using a DNN and Random Forest \cite{GarciaCeja2018_classification_bipolar}. 

\section{Use cases in the real world}



\section{Challenges and ethical concerns}
In most projects in the medical fields, there are going to be ethical concerns and challenges with privacy. What happens if someone unauthorized for the data gets access to it? What if the database gets hacked? With new regulations (GDPR), which means that users have the right to be forgotten or deleted. However, in this project, all data is anonymized (only referenced by an id), so there will be no persons mentioned. If the dataset were not to be anonymized, and the patient's names were in it, things could get problematic if it got into the wrong hands. 
