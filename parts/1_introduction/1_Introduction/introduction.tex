%%%%%% INTRODUCTION %%%%%%

\section{Motivation}

We are using Artificial Intelligence everywhere these days. Everyone is talking about how self-driving cars will change the world. Every smartphone company includes AI in their phones, striving to make the day of the user effortless. Thanks to Machine Learning or more specifically Deep Learning, AI can already beat humans in most games and do repetitive tasks for us. The concept of AI has been around for a long time (more than 80 years by among others Alan Turing \cite{turing1938}), but until 2012 Deep Learning was not a valid form of AI \cite{topol2019}. Today Deep Learning is widely used by the largest companies to learn all patterns possible about their customers. 

Research on the usage of AI in the medical field is particularly interesting, as the results can be life-changing. Instead of getting diagnosed only by a single doctor, a global AI will influence his or her decisions. The difference between independent doctors worldwide would not matter as much, and with high accuracy, doctors instantly would know a lot about the patient. 

[Stories from book]

Depression is what these people share. Three hundred and fifty million people globally are fighting this burden \cite{burden_of_depression}, which in a lot of the cases does not end well for them. AI has enormous potential here to help with diagnoses, help in the process of prescribing the right medicine to patients and predict that a person is likely to attempt suicide \cite{topol2019}. 

\section{Thesis Overview}
In \textbf{chapter 2} we provide background information about mental health and machine learning. In mental health, we describe the topics of bipolar disorder and how the Montgomery-Åsberg Depression Rating Scale can tell patients how depressed they are. Then we introduce important topics within machine learning, and how easy it is to get started writing models in Keras, a machine learning framework for Python. Related work is the last section in \textbf{chapter 2}, where we mention several papers related to the topics of this thesis: mental health and machine learning. \\

\noindent \textbf{Chapter 3} is where we describe the goals that we want to achieve in the thesis. We present the dataset we use to achieve those goals, and also how we structure and preprocess the data so that it can be used to train models. Performance metrics that we use later in order to evaluate a trained model is another topic we define in the chapter.\\

\noindent \textbf{Chapter 4} contains a description of the neural network model for each goal that we described in \textbf{chapter 3} in addition to a simple linear regression model. We describe all layers in the models and include the source code used to create them in the machine learning framework Keras. \\

\noindent In \textbf{chapter 5} we provide a walkthrough of how we trained the models in order to reach our goals, then evaluated them with metrics described in \textbf{chapter 3}. \\

\noindent \textbf{Chapter 6} is where we discuss whether convolutional neural networks can be used to classify and predict mental health issues. We also suggest real-world applications of our work, focusing on the pros and cons of creating and trusting an AI for mental health diagnoses. \\

\noindent \textbf{Chapter 7} contains our conclusion and a summary of the chapters, and we give directions for future research.